\documentclass[a4paper,12pt]{article}
\usepackage[russian]{babel} % (нельзя с polyglossia)
\usepackage{fontspec}
\setmainfont[Ligatures=TeX]{Times New Roman} % Шрифт для основного текста
\setsansfont[Ligatures=TeX]{Arial}
\setmonofont{Consolas} % Шрифт для кода
\usepackage{amsmath, amssymb} % Для математических формул
\usepackage[left=1.45cm, right=1.45cm, top=1.5cm, bottom=1.5cm]{geometry} % Настройка полей
\usepackage{graphicx}

\usepackage{xcolor}
\usepackage{enumitem}
\usepackage{ulem}
\usepackage{tcolorbox}

\usepackage{booktabs}

\AtBeginDocument{\boldmath} % Опционально: все мат.элементы жирным шрифтом

\usepackage{adjustbox}

\usepackage{float} % Для жесткой фиксации изображений [H]

\setlength{\parindent}{0pt} % Опционально: отрубаем по рофлу красную строку

\usepackage[colorlinks=true]{hyperref}
\hypersetup{
	colorlinks=true,
	linkcolor=blue, % Почему-то зависит от colorlinks=true
}

%-------------------РУССКАЯ НУМЕРАЦИЯ В ENUMERATE-------------------%
% Создаем команду для русских букв
\newcommand{\rusitem}[1]{
	\par % без этого почему-то первый пункт оставался на строке с текстом
	\stepcounter{ruscount}
	\noindent
	\ifcase\value{ruscount}\or а)\or б)\or в)\or г)\or д)\or е)\or ж)\or з)\or и)\or й)\or
	к)\or л)\or м)\or н)\or о)\or п)\or р)\or с)\or т)\or у)\or
	ф)\or х)\or ц)\or ч)\or ш)\or щ)\or ъ)\or ы)\or ь)\or э)\or ю)\or я)\fi
	\ #1\par
}

% Создаем новый счетчик
\newcounter{ruscount}

%-------------------ЦВЕТНОЙ ФОН ФЛОМАСТЕРА-------------------%
% Цвета
\definecolor{lightblue}{RGB}{202,244,255}
\definecolor{pink}{RGB}{255,230,245}

% Цветные боксы
\newtcolorbox{highlightbox}{
	colback=lightblue,
	colframe=lightblue,
	boxrule=0pt,
	arc=3mm,
	auto outer arc,
	boxsep=2mm,
	left=2mm, right=2mm, top=1mm, bottom=1mm
}

\newtcolorbox{pinkbox}{
	colback=pink,
	colframe=pink,
	boxrule=0pt,
	arc=2mm,
	auto outer arc,
	boxsep=1mm,
	left=1mm, right=1mm, top=0.5mm, bottom=0.5mm
}

%-------------------КРУГОВАЯ ОБВОДКА-------------------%
\usepackage{titlesec}
\usepackage{tikz}

% Универсальная команда для обводки
\newcommand{\circled}[1]{
	\tikz[baseline=(char.base)]{
		\node[draw,circle,inner sep=1pt,minimum size=1.2em](char){\strut #1};
	}
}

% Модифицированная команда для разделов с обведённым номером
\newcommand{\circledsection}[2][]{
	% \refstepcounter{section} % Раскомментировав - раздел будет считаться в общей нумерации
	\par\vspace{4ex plus 1ex minus .2ex}
	\noindent{\Large\bfseries\circled{\thesection}~#2\par} % \thesection передает номер из оглавление в раздел
	\addcontentsline{toc}{section}{\protect\numberline{\thesection}#2} % С #1 будет убрано название и останутся только точки
	\vspace{2.5ex plus .2ex}
}

% Модифицированная команда для подразделов с обведённым номером
\newcommand{\circledsubsection}[1]{
	\refstepcounter{subsection} % Отвечает за нуммерацию
	\par\vspace{3.25ex plus 1ex minus .2ex}
	\noindent{\large\bfseries\circled{\thesubsection}~#1\par}
	\addcontentsline{toc}{subsection}{\protect\numberline{\thesubsection}#1}
	\vspace{2.3ex plus .2ex}
}

% Стандартные подразделы (если раскомментить - subsection* (без нумерации) перестанет работать)
% \makeatletter
% \renewcommand{\subsection}{\@startsection{subsection}{2}{\z@}
%	{-3.25ex\@plus -1ex \@minus -.2ex}
%	{1.5ex \@plus .2ex}
%	{\normalfont\large\bfseries}
% }
% \makeatother

% Реализация списков со сквозной нумерацией (просто \item в них писать не надо)
\newcounter{flexcounter}
\newlist{flexlist}{enumerate}{3}
\setlist[flexlist]{
	before=\setcounter{flexcounter}{0},
	label={\arabic*.},
	ref=\arabic*,
	align=left,
	leftmargin=*,
	labelwidth=!,
	labelindent=0pt
}

\newcommand{\flexlabel}[1]{
	\ifnum\value{flexcounter}>0\relax
	\arabic{flexcounter}.
	\fi
}

\newcommand{\circleditem}{
	\refstepcounter{flexcounter}
	\item[\circled{\arabic{flexcounter}}]
}

\newcommand{\plainitem}{
	\refstepcounter{flexcounter}
	\item[\arabic{flexcounter}]
}

%-------------------ДЕЛАЕМ ТОЧКИ ОГЛАВЛЕНИЯ АКТИВНЫМИ-------------------%
% Переопределяем \@dottedtocline для полной ссылки
\makeatletter
\def\@dottedtocline#1#2#3#4#5{
	\ifnum #1>\c@tocdepth \else
	\vskip \z@ \@plus.2pt
	{\leftskip #2\relax
		\rightskip \@tocrmarg \parfillskip -\rightskip
		\parindent #2\relax\@afterindenttrue
		\interlinepenalty\@M
		\leavevmode
		\@tempdima #3\relax
		\begingroup
		\ifnum #1=1 \bfseries \fi % Жирный шрифт только для первого уровня
		\parindent \z@ \leftskip #3\relax
		% \advance\leftskip by 1.5em % - задаем сдвиг подзаголовка
		% \hskip -\leftskip % - двигаем подзаголовок влево
		\hyper@linkstart{link}{\Hy@tocdestname}
		#4 % текст заголовка
		\leaders\hbox{$\m@th
			\mkern \@dotsep mu\hbox{.}\mkern \@dotsep mu$}\hfill
		\nobreak\hb@xt@\@pnumwidth{\hfil #5} % номер страницы
		\hyper@linkend
		\endgroup
		\par}
	\fi}
% Переопределяем \l@section для использования \@dottedtocline
\renewcommand*\l@section[2]{
	\ifnum \c@tocdepth >\z@
	\addpenalty\@secpenalty
	\addvspace{1.0em \@plus\p@}
	\@dottedtocline{1}{0em}{1.5em}{#1}{#2}
	\fi}
\makeatother
